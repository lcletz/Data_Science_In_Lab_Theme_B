\documentclass{beamer}
\usepackage[utf8]{inputenc}
\usepackage[french]{babel}
\usepackage[T1]{fontenc}
\usepackage{amsmath}
\usepackage{amsfonts}
\usepackage{csquotes}
\usepackage{amssymb}
\usepackage{graphicx}

\usetheme{Szeged}
\usepackage{xcolor}
\definecolor{mygreen}{RGB}{46,139,87}
\setbeamercolor{structure}{fg=mygreen}
\usefonttheme{structuresmallcapsserif}

\setbeamertemplate{footline}[frame number]

\title{Plan d'analyse statistique}
\subtitle{BICAR-ICU - Thème B}
\author{ACHIQ Aya, CLETZ Laura, ZHU Qingjian}

\date{\footnotesize 19 Novembre 2025}

\titlegraphic{ 
\centering
\includegraphics[height=1.2cm]{UM.png}\hspace{0.3cm}%
\includegraphics[height=1.2cm]{BioSSD_logo.png}\hspace{0.1cm}%
\includegraphics[height=1.2cm]{FdS.jpg}
}

\begin{document}

\begin{frame}

  \titlepage

\end{frame}

\section{Problématique}

\begin{frame}{Problématique}

  \centering
  Critère de jugement secondaire :\\
  \Large\textbf{{Un traitement au bicarbonate de sodium (BS) réduit-il la dépendance à la dialyse chez les patients en soins intensifs ?}}\vspace{1cm}

  \small{Comment gérer les données manquantes MAR (Missing At Random) dans cette analyse ?}

\end{frame}

\section{Imputation des données}

\begin{frame}{Imputation des données}

  \textbf{Méthode :} \textit{Multivariate Imputation by Chained Equations (MICE)}\\

  \vspace{0.5cm}
  \begin{itemize}
    \item Spécialement adaptée aux données \textbf{MAR (Missing At Random)}
    \item Compatible avec des \textbf{variables mixtes} (binaires, continues, catégorielles)
    \item Imputation \textbf{multiple} : reflète l’incertitude liée aux valeurs manquantes
    \item Utilisation dans les essais cliniques validée par FDA/EMA
  \end{itemize}
  \vfill
  \tiny{A. D. Shah, J. W. Bartlett, J. Carpenter, O. Nicholas, H. Hemingway (2014) \textbf{Comparison of random forest and parametric imputation models for imputing missing data using MICE: a CALIBER study.} \textit{American journal of epidemiology}, https://doi.org/10.1093/aje/kwt312}
  \\\vspace{0.1cm}
  \tiny{S. van Buuren (2018) \textbf{Flexible Imputation of Missing Data, Second Edition.} \textit{Chapman and Hall/CRC}, https://doi.org/10.1201/9780429492259}

\end{frame}

\section{Méthode d'analyse}

\begin{frame}{Méthode d’analyse selon la définition de Y}

\small
\begin{tabular}{|p{1.7cm}|p{4.8cm}|p{3.5cm}|}
\hline
\textbf{Type de Y} & \textbf{Définition} & \textbf{Méthode d’analyse} \\
\hline
Binaire & RRT administrée au moins une fois entre J0–J28 (oui/non) & Régression logistique \\
\hline
Comptage  & Nombre total de jours avec RRT (valeurs 0 à 28) & Régression de Poisson  \\
\hline
Temporelle  & Temps (en jours) avant la première séance de RRT ou jusqu’à censure (J28 ou décès) & Modèle de Cox (analyse de survie) \\
\hline
\end{tabular}

\vspace{0.3cm}
\small{Toutes les méthodes permettent d’estimer l’effet du traitement (BS vs no-BS) sur la dépendance à la dialyse.}

\end{frame}

\section{Variables retenues}

\begin{frame}{Variables retenues}

  \begin{itemize}
    \item \textbf{Critère de jugement secondaire :} Y à définir
    \item \textbf{Variable principale d'intérêt :} bras/groupe de traitement (BS vs no-BS)
    \item \textbf{Covariables potentielles :}
    \begin{itemize}
      \item Paramètres démographiques (âge)
      \item Paramètres cliniques à l'inclusion/randomisation (score SOFA et AKIN)
      \item Paramètres biologiques (créatinine, urée)
      \item Comorbidités (maladie rénale chronique)
    \end{itemize}
  \end{itemize}
  
\end{frame}

\end{document}